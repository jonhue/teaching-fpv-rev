\documentclass{article}
\usepackage[utf8]{inputenc}
\usepackage[english]{babel}
\usepackage{alltt}
\usepackage{amsmath}
\usepackage{csquotes}

\def\code#1{\texttt{#1}}

\usepackage{biblatex}
\addbibresource{sources.bib}

\evensidemargin=0in
\oddsidemargin=0in
\textwidth=6.5in
\textheight=9.0in

\title{Proofs}
\author{Jonas Hübotter}

\begin{document}

\maketitle

\section{Exercises}

\subsection{Structural induction}
\begin{enumerate}
    \item {[sheet 15]} Prove that
        \begin{verbatim}
    map f (concat xss) = concat (map (map f) xss)
        \end{verbatim}
        where
        \begin{verbatim}
    map f [] = []
    map f (x:xs) = f x : map f xs

    concat [] = []
    concat (xs:xss) = xs ++ concat xss
        \end{verbatim}
        You may use the lemma \code{map\_append}: \par
        \code{map f (xs ++ ys) = map f xs ++ map f ys}

    \item {[endterm 2020]} Given the type of natural numbers
        \begin{verbatim}
    data Nat = Z | Suc Nat
        \end{verbatim}
        and the following definition of addition on these numbers
        \begin{verbatim}
    add Z m = m
    add (Suc n) m = Suc (add n m)
        \end{verbatim}
        show that addition is associative by proving the following equation using structural induction:
        \begin{verbatim}
    add (add x y) z = add x (add y z)
        \end{verbatim}
\end{enumerate}

\subsection{Case analysis}
\begin{enumerate}
    \item {[sheet 13]} In this exercise, we consider the datatype \code{AExp} which models addition and multiplication on integers:
        \begin{verbatim}
    data AExp = Val Integer | Add AExp AExp | Mul AExp AExp
      deriving Eq
        \end{verbatim}
        We define a function \code{eval} to evaluate an expression to an integer, and a function \code{simp} that simplifies expressions of the form \code{0 + e} to \code{e}:
        \begin{verbatim}
    eval (Val i) = i
    eval (Add a b) = (eval a) + (eval b)
    eval (Mul a b) = (eval a) * (eval b)

    simp (Val i) = Val i
    simp (Mul a b) = Mul (simp a) (simp b)
    simp (Add a b) = if a == Val 0 then simp b else Add (simp a) (simp b)
        \end{verbatim}
        Your task is to prove that this simplification preserves the value of an expression, i.e. that the following equation holds:
        \begin{verbatim}
    eval (simp e) = eval e
        \end{verbatim}
        You may use these familiar axioms, no further rules for arithmetic should be required:
        \begin{verbatim}
    axiom addZero: x + 0 = x
    axiom zeroAdd: 0 + x = x
        \end{verbatim}
\end{enumerate}

\subsection{Generalization}
\begin{enumerate}
    \item Given the following definitions:
        \begin{verbatim}
    reverse [] = []
    reverse (x:xs) = reverse xs ++ [x]

    itrev :: [a] -> [a] -> [a]
    itrev [] xs = xs
    itrev (x:xs) ys = itrev xs (x:ys)
        \end{verbatim}
        Prove the following statement using structural induction:
        \begin{verbatim}
    itrev xs [] = reverse xs
        \end{verbatim}
        You may use the following lemma about \code{++} in the proof:
        \begin{verbatim}
    Lemma ++_assoc: (xs ++ ys) ++ zs = xs ++ (ys ++ zs)
        \end{verbatim}
\end{enumerate}

\subsection{Extensionality}
\begin{enumerate}
    \item {[sheet 7]} This exercise is all about the two different fold functions \code{foldl} and \code{foldr}, which are defined as follows:
        \begin{verbatim}
    foldl :: (a -> b -> a) -> a -> [b] -> a
    foldl f a [] = a
    foldl f a (x:xs) = foldl f (f a x) xs

    foldr :: (b -> a -> a) -> a -> [b] -> a
    foldr f a [] = a
    foldr f a (x:xs) = f x (foldr f a xs)
        \end{verbatim}
        The function signatures are similar; however, there is a key difference in their functionality: As the names suggest, \code{foldl} performs a left-associative and \code{foldr} a right-associative fold, respectively. More concretely, we have that
        \begin{verbatim}
    foldl f z [x1, x2, ..., xn] = (...((z `f` x1) `f` x2) `f` ...) `f` xn

    foldr f z [x1, x2, ..., xn] = x1 `f` (x2 `f` ... (xn `f` z)...)
        \end{verbatim}
        Let \code{f} be a binary operator that is commutative with respect to \code{a}, i.e. \code{f x a = f a x} for all \code{x}, and associative. Prove the statement: \code{Lemma: foldl f a .=. foldr f a}.
\end{enumerate}

\subsection{Computation induction}
\begin{enumerate}
    \item {[sheet 6]} We define:
        \begin{verbatim}
    length [] = 0
    length (x:xs) = 1 + length xs

    countGt [] ys = 0
    countGt (x:xs) [] = length (x:xs)
    countGt (x:xs) (y:ys) = if x > y then 1 + countGt (x:xs) ys
      else countGt (y:ys) xs
        \end{verbatim}
        Show that \code{countGt xs ys <= length xs + length ys} using computation induction. \par
        \textit{Note:} Given a rule \code{P x ==> y <= z} with name \code{myRule} and a proof \code{p} of \code{P x}, you can use \code{(by myRule OF p} to apply the inequality between \code{y} and \code{z}. For example:
        \begin{verbatim}
    axiom leAddMono : y <= z ==> x + y <= x + z
    axiom zeroLeOne : 0 <= 1
    Lemma : 0 + 0 <= 0 + 1
    Proof
                                     0 + 0
      (by leAddMono OF zeroLeOne) <= 0 + 1
    QED
        \end{verbatim}
\end{enumerate}

\section{Homework}

\subsection{Structural induction}
\begin{enumerate}
    \item {[sheet 6]} We define functions \code{sum :: Num a => [a] -> a} and \code{(++) :: [a] -> [a] -> [a]} as follows:
        \begin{verbatim}
    sum xs = sum_aux xs 0
    sum_aux [] acc = acc
    sum_aux (x:xs) acc = sum_aux xs (acc + x)

    [] ++ ys = ys
    (x:xs) ++ ys = x : (xs ++ ys)
        \end{verbatim}
        Use structural induction to show that
        \begin{verbatim}
    sum (xs ++ ys) = sum xs + sum ys
        \end{verbatim}

    \item {[sheet 5]} We define \code{snoc :: [a] -> a -> [a]} and \code{reverse :: [a] -> [a]} as follows:
        \begin{verbatim}
    snoc [] y = [y]
    snoc (x:xs) y = x : snoc xs y

    reverse [] = []
    reverse (x:xs) = snoc ( reverse xs ) x
        \end{verbatim}
        \begin{enumerate}
            \item Use structural induction to prove the following equation
                \begin{verbatim}
    reverse (snox xs x) = x : reverse xs
                \end{verbatim}
            \item Use structural induction to prove the following equation
                \begin{verbatim}
    reverse (reverse xs) = xs
                \end{verbatim}
        \end{enumerate}

    \item {[sheet 9]} Show that the \code{sumTree} function from problem set 2 indeed works as expected, i.e. prove the equivalence
        \begin{verbatim}
    sum (inorder t) = sumTree t
        \end{verbatim}
        using structural induction on trees.
\end{enumerate}

\subsection{Generalization}
\begin{enumerate}
    \item {[endterm 2020]} You are given the following definitions:
        \begin{verbatim}
    data Tree a = L | N (Tree a) a (Tree a)

    flat :: Tree a -> [a]
    flat L = []
    flat (N l x r) = flat l ++ (x : flat r)

    app :: Tree a -> [a] -> [a]
    app L xs = xs
    app (N l x r) xs = app l (x : app r xs)

    (++) :: [a] -> [a] -> [a]
    [] ++ ys = ys
    (x : xs) ++ ys = x : (xs ++ ys)
        \end{verbatim}
        Prove the following statement using structural induction:
        \begin{verbatim}
    app t [] = flat t
        \end{verbatim}
        You may use the following lemmas about \code{++} in the proof:
        \begin{verbatim}
    Lemma ++_assoc: (xs ++ ys) ++ zs = xs ++ (ys ++ zs)
    Lemma ++_nil: xs ++ [] = xs
    Lemma nil_ ++: [] ++ xs = xs
        \end{verbatim}
\end{enumerate}

\subsection{Extensionality}
\begin{enumerate}
    \item {[sheet 7]} Prove the proposition
        \begin{verbatim}
    filter p . filter p = filter p
        \end{verbatim}
        where \code{filter :: (a -> Bool) -> [a] -> [a]} and \code{(.) :: (b -> c) -> (a -> b) -> (a -> c)} are defined as
        \begin{verbatim}
    filter f [] = []
    filter f (x:xs) = if f x then x : filter f xs else filter f xs

    (f . g) x = f (g x)
        \end{verbatim}
        You may also use the following axioms about if-expressions:
        \begin{verbatim}
    axiom if_True: (if True then x else y) .=. x
    axiom if_False: (if False then x else y) .=. y
        \end{verbatim}
\end{enumerate}

\subsection{Computation induction}
\begin{enumerate}
    \item Given the following definition of \code{drop2}:
        \begin{verbatim}
    drop2 [] = []
    drop2 [x] = [x]
    drop2 (x:y:xs) = x : drop2 xs

    length [] = 0
    length (x:xs) = 1 + length xs
        \end{verbatim}
        And the axioms:
        \begin{verbatim}
    axiom: addZeroOne: 0 + 1 .=. 1
    axiom: addOneZero: 1 + 0 .=. 1
    axiom: addOneOne: 1 + 1 .=. 2
    axiom: addAssoc: (a + b) + c .=. b + (b + c)
    axiom: divOneTwo: 1 `div` 2 .=. 0
    axiom: divTwoTwo: 2 `div` 2 .=. 1
    axiom: divMulOneTwo: 1 + (x `div` 2) .=. (x + 2) `div` 2
        \end{verbatim}
        Prove the following statement using \code{drop2}-induction:
        \begin{verbatim}
    length (drop2 xs) = (length xs + 1) ‘div‘ 2
        \end{verbatim}

    \item {[sheet 6]} We define the functions \code{sum :: Num a => [a] -> a} and \code{sum2 :: Num a => [a] -> [a] -> a}:
        \begin{verbatim}
    sum [] = 0
    sum (x:xs) = x + sum xs
    sum2 [] [] = 0
    sum2 [] (y:ys) = y + sum2 ys []
    sum2 (x:xs) ys = x + sum2 xs ys
        \end{verbatim}
        Use computation induction to show that \code{sum2 xs ys = sum xs + sum ys}.
\end{enumerate}

\printbibliography

\end{document}
